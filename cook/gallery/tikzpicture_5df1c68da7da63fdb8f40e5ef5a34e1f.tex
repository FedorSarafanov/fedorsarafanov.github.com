\documentclass[tikz]{standalone}\usepackage{cmap}
\usepackage[T2A]{fontenc}
\usepackage[utf8x]{inputenc}

\usepackage[mode=buildnew]{standalone}

\usepackage
	{
		amssymb,
		% misccorr,
		amsfonts,
		amsmath,
		amsthm,
		wrapfig,
		makecell,
		multirow,
		indentfirst,
		ulem,
		graphicx,
		subcaption,
		float,
		tikz,
		caption,
		tikz-3dplot,
		csvsimple,
		color,
		booktabs,
		pgfplots,
		pgfplotstable,
		fancyhdr,
	}  


\usepackage[outline]{contour}

\usepackage[europeanresistors,americaninductors]{circuitikz}
\tikzset{
  pics/carc/.style args={#1:#2:#3}{
    code={
      \draw[pic actions] (#1:#3) arc(#1:#2:#3);
    }
  }
}

\linespread{1.3} 
\frenchspacing 

 
\usetikzlibrary
	{
		decorations.pathreplacing,
		decorations.pathmorphing,
		patterns,
		calc,
		scopes,
		arrows,
		fadings,
		through,
		shapes.misc,
		arrows.meta,
		3d,
		quotes,
		angles,
		babel
	}


\tikzset{
	force/.style=	{
		>=latex,
		draw=blue,
		fill=blue,
				 	}, 
	%				 	
	axis/.style=	{
		densely dashed,
		gray,
		font=\small,
					},
	%
	acceleration/.style={
		>=open triangle 60,
		draw=magenta,
		fill=magenta,
					},
	%
	inforce/.style=	{
		force,
		double equal sign distance=2pt,
					},
	%
	interface/.style={
		pattern = north east lines, 
		draw    = none, 
		pattern color=gray!60,
					},
	cross/.style=	{
		cross out, 
		draw=black, 
		minimum size=2*(#1-\pgflinewidth), 
		inner sep=0pt, outer sep=0pt,
					},
	%
	cargo/.style=	{
		rectangle, 
		fill=black!70, 
		inner sep=2.5mm,
					},
	%
	caption/.style= {
		midway,
		fill=white!20, 
		opacity=0.9
					},
	%
	}


\newcommand{\irodov}[1]{\geometry{left=1cm,right=1cm,top=2cm,bottom=1.5cm,bindingoffset=0cm}
\pagestyle{fancy}\fancyhead{}\fancyhead[C]{МКТ}\fancyhead[R]{Сарафанов Ф.Г.}  \fancyhead[L]{Иродов -- №#1}\fancyfoot{} \fancyfoot[C]{\thepage}}
\newcommand{\yakovlev}[1]{\geometry{left=1cm,right=1cm,top=2cm,bottom=1.5cm,bindingoffset=0cm}
\pagestyle{fancy}\fancyhead{}\fancyhead[C]{МКТ}\fancyhead[R]{Сарафанов Ф.Г.}  \fancyhead[L]{Яковлев -- №#1}\fancyfoot{} \fancyfoot[C]{\thepage}}
\newcommand{\wrote}[1]{\geometry{left=1cm,right=1cm,top=2cm,bottom=1.5cm,bindingoffset=0cm}
\pagestyle{fancy}\fancyhead{}\fancyhead[C]{МКТ}\fancyhead[R]{Сарафанов Ф.Г.} \fancyhead[L]{Под запись -- <<#1>>}\fancyfoot{} \fancyfoot[C]{\thepage}}

\newcommand{\siv}[1]{\geometry{left=1cm,right=1cm,top=2cm,bottom=1.5cm,bindingoffset=0cm}
\pagestyle{fancy}\fancyhead{}\fancyhead[C]{МКТ}\fancyhead[R]{Сарафанов Ф.Г.} \fancyhead[L]{Сивухин -- №#1}\fancyfoot{} \fancyfoot[C]{\thepage}}




\newcommand{\irodovT}[1]{\geometry{left=1cm,right=1cm,top=2cm,bottom=1.5cm,bindingoffset=0cm}
\pagestyle{fancy}\fancyhead{}\fancyhead[C]{ТД}\fancyhead[R]{Сарафанов Ф.Г.}  \fancyhead[L]{Иродов -- №#1}\fancyfoot{} \fancyfoot[C]{\thepage}}
\newcommand{\yakovlevT}[1]{\geometry{left=1cm,right=1cm,top=2cm,bottom=1.5cm,bindingoffset=0cm}
\pagestyle{fancy}\fancyhead{}\fancyhead[C]{ТД}\fancyhead[R]{Сарафанов Ф.Г.}  \fancyhead[L]{Яковлев -- №#1}\fancyfoot{} \fancyfoot[C]{\thepage}}
\newcommand{\wroteT}[1]{\geometry{left=1cm,right=1cm,top=2cm,bottom=1.5cm,bindingoffset=0cm}
\pagestyle{fancy}\fancyhead{}\fancyhead[C]{ТД}\fancyhead[R]{Сарафанов Ф.Г.} \fancyhead[L]{Под запись -- <<#1>>}\fancyfoot{} \fancyfoot[C]{\thepage}}

\newcommand{\sivT}[1]{\geometry{left=1cm,right=1cm,top=2cm,bottom=1.5cm,bindingoffset=0cm}
\pagestyle{fancy}\fancyhead{}\fancyhead[C]{ТД}\fancyhead[R]{Сарафанов Ф.Г.} \fancyhead[L]{Сивухин -- №#1}\fancyfoot{} \fancyfoot[C]{\thepage}}

\newenvironment{tikzpict}
    {
	    \begin{figure}[htbp]
		\centering
		\begin{tikzpicture}
    }
    { 
		\end{tikzpicture}
		% \caption{caption}
		% \label{fig:label}
		\end{figure}
    }

\newcommand{\vbLabel}[3]{\draw ($(#1,#2)+(0,5pt)$) -- ($(#1,#2)-(0,5pt)$) node[below]{#3}}
\newcommand{\vaLabel}[3]{\draw ($(#1,#2)+(0,5pt)$) node[above]{#3} -- ($(#1,#2)-(0,5pt)$) }

\newcommand{\hrLabel}[3]{\draw ($(#1,#2)+(5pt,0)$) -- ($(#1,#2)-(5pt,0)$) node[right, xshift=1em]{#3}}
\newcommand{\hlLabel}[3]{\draw ($(#1,#2)+(5pt,0)$) node[left, xshift=-1em]{#3} -- ($(#1,#2)-(5pt,0)$) }



\newcommand\zi{^{\,*}_i}
\newcommand\sumn{\sum_{i=1}^{N}}


\tikzset{
	coordsys/.style={scale=1.8,x={(1.1cm,-0cm)},y={(0.5cm,1cm)}, z={(0cm,0.8cm)}},
	% coordsys/.style={scale=1.5,x={(0cm,0cm)},y={(1cm,0cm)}, z={(0cm,1cm)}}, 
	% coordsys/.style={scale=1.5,x={(1cm,0cm)},y={(0cm,1cm)}, z={(0cm,0cm)}}, 
}
\usepgfplotslibrary{units}
%Русский язык в input (если расположить раньше, будет ошибка)
% \makeatletter
%     \let\old@input\input
%     \renewcommand\input[1]{%
%         \expandafter\old@input{\detokenize{#1}}%
%     }
% \makeatother

% \usepackage{import}

% \input{\source/tikz_line-annotation.tex}

\tikzset{
  pics/carc/.style args={#1:#2:#3}{
    code={
      \draw[pic actions] (#1:#3) arc(#1:#2:#3);
    }
  },
  dash/.style={
  	dash pattern=on 5mm off 5mm
  }
}

\newcommand{\mean}[1]{\langle#1\rangle}

% Draw line annotation
% Input:
%   #1 Line offset (optional)
%   #2 Line angle
%   #3 Line length
%   #5 Line label
% Example:
%   \lineann[1]{30}{2}{$L_1$}
\newcommand{\lineann}[4][0.5]{%
    \begin{scope}[rotate=#2, blue,inner sep=2pt, ]
        \draw[dashed, blue!40] (0,0) -- +(0,#1)
            node [coordinate, near end] (a) {};
        \draw[dashed, blue!40] (#3,0) -- +(0,#1)
            node [coordinate, near end] (b) {};
        \draw[|<->|] (a) -- node[fill=white, scale=0.8] {#4} (b);
    \end{scope}
}


\pgfplotsset{
    % most recent feature set of pgfplots
    compat=newest,
}

\newcommand\ct[1]{\text{\rmfamily\upshape #1}}
% \usepackage[europeanresistors,americaninductors]{circuitikz}

\newcommand*{\const}{\ct{const}}
% Style to select only points from #1 to #2 (inclusive)
\pgfplotsset{select/.style 2 args={
    x filter/.code={
        \ifnum\coordindex<#1\def\pgfmathresult{}\fi
        \ifnum\coordindex>#2\def\pgfmathresult{}\fi
    }
}}
\usepackage{array}\begin{document}\begin{tikzpicture}[
    force/.style={>=latex,draw=blue,fill=blue},
    % axis/.style={densely dashed,gray,font=\small},
    axis/.style={densely dashed,black!60,font=\small},
    M/.style={rectangle,draw,fill=lightgray,minimum size=0.5cm,thin},
    m2/.style={draw=black!30, rectangle,draw,thin, fill=blue!2, minimum width=0.7cm,minimum height=0.7cm},
    m1/.style={draw=black!30, rectangle,draw,thin, fill=blue!2, minimum width=0.7cm,minimum height=0.7cm},
    plane/.style={draw=black!30, very thick, fill=blue!5, line width=1pt},
    % base/.style={draw=black!70, very thick, fill=blue!4, line width=2pt},
    string/.style={draw=black, thick},
    pulley/.style={thick},
    interface/.style={draw=gray!60,
        % The border decoration is a path replacing decorator. 
        % For the interface style we want to draw the original path.
        % The postaction option is therefore used to ensure that the
        % border decoration is drawn *after* the original path.
        postaction={draw=gray!60,decorate,decoration={border,angle=-135,
                    amplitude=0.3cm,segment length=2mm}}},
]
\def\height{1.5cm}
\def\width{3cm}
\def\Ti{\vec{T}_1}
\def\Tii{\vec{T}_2}
\def\FINii{\vec{F}_{in\,2}^\text{пост}}
\def\FINi{\vec{F}_{in\,1}^\text{пост}}
\def\radius{0.35cm}    
\matrix[column sep=1cm, row sep=0cm] {

% \draw[use as bounding box] (0,0) rectangle (1,1);

\begin{scope}%
            [scale=1.2, transform shape]
    \coordinate (base left) at (-\width/2,-\height/2);
    \coordinate (top left) at (-\width/2,\height/2);
    \coordinate (top right) at (\width/2,\height/2);
    \coordinate (base right) at (\width/2,-\height/2);
    \coordinate (top mid) at ($(top left)!0.5!(top right)$);

    \draw[axis,->] (base left)  -- ($(base right)+(0.3,0)$) node[right] {$+y'$};
    \draw[axis,->] (base left)  -- ($(top left)+(0,0.3)$) node[right] {$+x'$};

    \draw[plane]  (base left) -- (base right) --  (top right)  -- (top left) -- cycle;
    \draw[interface]  ($(base left)-(0.3,0)$) -- ($(base right)+(0.3,0)$);

    \path (top mid) node[m1, yshift=0.35cm] (m1) {$m_1$};

    \path ($(base left)!0.5!(top right)$) node[yshift=0cm] (A) {$A$};

    \draw[->,>=open triangle 60] (0,2) -- node[above,pos=0.5] {$\vec{a_0}$} (1,2);

    \draw[pulley] ($(top right)-(0,\radius)$) arc (-90:180:\radius) coordinate (pulley);
    \filldraw[fill=black] (top right) circle(0.05cm);

    \draw[string] (m1.east) -- ++(1.15,0) arc (90:0:\radius)
                  -- ++(0,-1) node[m2] {$m_2$};       
\end{scope} 
&
\begin{scope}[scale=1.2, transform shape]
    
   \node[m1,transform shape] (m1) {};

    {[axis,->]
        \draw (0,-1) -- (0,2) node[right] {$+y'$};
        \draw (m1.south) -- ++(2,0) node[right] {$+x'$};
    }

    % Forces
    {
        \draw[force,->] (m1.south) -- ++(0,1.5) node[above right] {$\vec{N}_1$};
        \draw[force,->](m1.south) -- ++(-1,0) node[below] {$\vec{f}_{R_{1A}}$};
        \draw[force,->] (m1.east) -- ++(1,0) node[above] {$\vec{T}_1$};
        \draw[force,double equal sign distance=2pt,->] (m1.west) -- ++(-1,0) node[above] {$\FINi$};
    }

    \draw[force,->] (m1.center) -- ++(0,-1) node[below] {$m_1\vec{g}$};
 \end{scope}
\\
\node[draw=none,text width=5cm, line width=0mm] at (0,0.5) {
Возьмем НИСО, связанную с бруском $A$. 
\begin{gather}
    \nonumber \text{Нить, блок идеальные. Тогда:}\\
    T_1=T_2\\
    \nonumber \text{Cилы инерции:}\\
    \FINi=-m_1\vec{a}_0\\
    \FINii=-m_2\vec{a}_0
\end{gather}

};
&
\begin{scope}[scale=1.2, transform shape]

    \node[m2,transform shape] (m2) {};
    % Draw axes and help lines

    {[axis,->]
        \draw (0,-1) -- (0,2) node[right] {$+y'$};
        \draw (m2.south) -- ++(2,0) node[right] {$+x'$};
    }

    % Forces
    {
        \draw[force,->] (m2.center) -- ++(0,1) node[above right] {$\vec{T}_2$};
        \draw[force,->](m2.west) -- ++(0,1) node[left] {$\vec{f}_{R_{2A}}$};
        \draw[force,->](m2.west) -- ++(1.5,0) node[right] {$\vec{N}_2$};
        \draw[force,double equal sign distance=2pt,->] (m2.west) -- ++(-1,0) node[below] {$\FINii$};
        \draw[force,->] (m2.center) -- ++(0,-1) node[below] {$m_2\vec{g}$};
    }
 \end{scope}
 \\
 %
};

\end{tikzpicture}\end{document}