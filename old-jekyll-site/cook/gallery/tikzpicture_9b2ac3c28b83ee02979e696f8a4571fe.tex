\documentclass[tikz]{standalone}\usepackage{cmap}
\usepackage[T2A]{fontenc}
\usepackage[utf8x]{inputenc}

\usepackage[mode=buildnew]{standalone}

\usepackage
	{
		amssymb,
		% misccorr,
		amsfonts,
		amsmath,
		amsthm,
		wrapfig,
		makecell,
		multirow,
		indentfirst,
		ulem,
		graphicx,
		subcaption,
		float,
		tikz,
		caption,
		tikz-3dplot,
		csvsimple,
		color,
		booktabs,
		pgfplots,
		pgfplotstable,
		fancyhdr,
	}  


\usepackage[outline]{contour}

\usepackage[europeanresistors,americaninductors]{circuitikz}
\tikzset{
  pics/carc/.style args={#1:#2:#3}{
    code={
      \draw[pic actions] (#1:#3) arc(#1:#2:#3);
    }
  }
}

\linespread{1.3} 
\frenchspacing 

 
\usetikzlibrary
	{
		decorations.pathreplacing,
		decorations.pathmorphing,
		patterns,
		calc,
		scopes,
		arrows,
		fadings,
		through,
		shapes.misc,
		arrows.meta,
		3d,
		quotes,
		angles,
		babel
	}


\tikzset{
	force/.style=	{
		>=latex,
		draw=blue,
		fill=blue,
				 	}, 
	%				 	
	axis/.style=	{
		densely dashed,
		gray,
		font=\small,
					},
	%
	acceleration/.style={
		>=open triangle 60,
		draw=magenta,
		fill=magenta,
					},
	%
	inforce/.style=	{
		force,
		double equal sign distance=2pt,
					},
	%
	interface/.style={
		pattern = north east lines, 
		draw    = none, 
		pattern color=gray!60,
					},
	cross/.style=	{
		cross out, 
		draw=black, 
		minimum size=2*(#1-\pgflinewidth), 
		inner sep=0pt, outer sep=0pt,
					},
	%
	cargo/.style=	{
		rectangle, 
		fill=black!70, 
		inner sep=2.5mm,
					},
	%
	caption/.style= {
		midway,
		fill=white!20, 
		opacity=0.9
					},
	%
	}


\newcommand{\irodov}[1]{\geometry{left=1cm,right=1cm,top=2cm,bottom=1.5cm,bindingoffset=0cm}
\pagestyle{fancy}\fancyhead{}\fancyhead[C]{МКТ}\fancyhead[R]{Сарафанов Ф.Г.}  \fancyhead[L]{Иродов -- №#1}\fancyfoot{} \fancyfoot[C]{\thepage}}
\newcommand{\yakovlev}[1]{\geometry{left=1cm,right=1cm,top=2cm,bottom=1.5cm,bindingoffset=0cm}
\pagestyle{fancy}\fancyhead{}\fancyhead[C]{МКТ}\fancyhead[R]{Сарафанов Ф.Г.}  \fancyhead[L]{Яковлев -- №#1}\fancyfoot{} \fancyfoot[C]{\thepage}}
\newcommand{\wrote}[1]{\geometry{left=1cm,right=1cm,top=2cm,bottom=1.5cm,bindingoffset=0cm}
\pagestyle{fancy}\fancyhead{}\fancyhead[C]{МКТ}\fancyhead[R]{Сарафанов Ф.Г.} \fancyhead[L]{Под запись -- <<#1>>}\fancyfoot{} \fancyfoot[C]{\thepage}}

\newcommand{\siv}[1]{\geometry{left=1cm,right=1cm,top=2cm,bottom=1.5cm,bindingoffset=0cm}
\pagestyle{fancy}\fancyhead{}\fancyhead[C]{МКТ}\fancyhead[R]{Сарафанов Ф.Г.} \fancyhead[L]{Сивухин -- №#1}\fancyfoot{} \fancyfoot[C]{\thepage}}




\newcommand{\irodovT}[1]{\geometry{left=1cm,right=1cm,top=2cm,bottom=1.5cm,bindingoffset=0cm}
\pagestyle{fancy}\fancyhead{}\fancyhead[C]{ТД}\fancyhead[R]{Сарафанов Ф.Г.}  \fancyhead[L]{Иродов -- №#1}\fancyfoot{} \fancyfoot[C]{\thepage}}
\newcommand{\yakovlevT}[1]{\geometry{left=1cm,right=1cm,top=2cm,bottom=1.5cm,bindingoffset=0cm}
\pagestyle{fancy}\fancyhead{}\fancyhead[C]{ТД}\fancyhead[R]{Сарафанов Ф.Г.}  \fancyhead[L]{Яковлев -- №#1}\fancyfoot{} \fancyfoot[C]{\thepage}}
\newcommand{\wroteT}[1]{\geometry{left=1cm,right=1cm,top=2cm,bottom=1.5cm,bindingoffset=0cm}
\pagestyle{fancy}\fancyhead{}\fancyhead[C]{ТД}\fancyhead[R]{Сарафанов Ф.Г.} \fancyhead[L]{Под запись -- <<#1>>}\fancyfoot{} \fancyfoot[C]{\thepage}}

\newcommand{\sivT}[1]{\geometry{left=1cm,right=1cm,top=2cm,bottom=1.5cm,bindingoffset=0cm}
\pagestyle{fancy}\fancyhead{}\fancyhead[C]{ТД}\fancyhead[R]{Сарафанов Ф.Г.} \fancyhead[L]{Сивухин -- №#1}\fancyfoot{} \fancyfoot[C]{\thepage}}

\newenvironment{tikzpict}
    {
	    \begin{figure}[htbp]
		\centering
		\begin{tikzpicture}
    }
    { 
		\end{tikzpicture}
		% \caption{caption}
		% \label{fig:label}
		\end{figure}
    }

\newcommand{\vbLabel}[3]{\draw ($(#1,#2)+(0,5pt)$) -- ($(#1,#2)-(0,5pt)$) node[below]{#3}}
\newcommand{\vaLabel}[3]{\draw ($(#1,#2)+(0,5pt)$) node[above]{#3} -- ($(#1,#2)-(0,5pt)$) }

\newcommand{\hrLabel}[3]{\draw ($(#1,#2)+(5pt,0)$) -- ($(#1,#2)-(5pt,0)$) node[right, xshift=1em]{#3}}
\newcommand{\hlLabel}[3]{\draw ($(#1,#2)+(5pt,0)$) node[left, xshift=-1em]{#3} -- ($(#1,#2)-(5pt,0)$) }



\newcommand\zi{^{\,*}_i}
\newcommand\sumn{\sum_{i=1}^{N}}


\tikzset{
	coordsys/.style={scale=1.8,x={(1.1cm,-0cm)},y={(0.5cm,1cm)}, z={(0cm,0.8cm)}},
	% coordsys/.style={scale=1.5,x={(0cm,0cm)},y={(1cm,0cm)}, z={(0cm,1cm)}}, 
	% coordsys/.style={scale=1.5,x={(1cm,0cm)},y={(0cm,1cm)}, z={(0cm,0cm)}}, 
}
\usepgfplotslibrary{units}
%Русский язык в input (если расположить раньше, будет ошибка)
% \makeatletter
%     \let\old@input\input
%     \renewcommand\input[1]{%
%         \expandafter\old@input{\detokenize{#1}}%
%     }
% \makeatother

% \usepackage{import}

% \input{\source/tikz_line-annotation.tex}

\tikzset{
  pics/carc/.style args={#1:#2:#3}{
    code={
      \draw[pic actions] (#1:#3) arc(#1:#2:#3);
    }
  },
  dash/.style={
  	dash pattern=on 5mm off 5mm
  }
}

\newcommand{\mean}[1]{\langle#1\rangle}

% Draw line annotation
% Input:
%   #1 Line offset (optional)
%   #2 Line angle
%   #3 Line length
%   #5 Line label
% Example:
%   \lineann[1]{30}{2}{$L_1$}
\newcommand{\lineann}[4][0.5]{%
    \begin{scope}[rotate=#2, blue,inner sep=2pt, ]
        \draw[dashed, blue!40] (0,0) -- +(0,#1)
            node [coordinate, near end] (a) {};
        \draw[dashed, blue!40] (#3,0) -- +(0,#1)
            node [coordinate, near end] (b) {};
        \draw[|<->|] (a) -- node[fill=white, scale=0.8] {#4} (b);
    \end{scope}
}


\pgfplotsset{
    % most recent feature set of pgfplots
    compat=newest,
}

\newcommand\ct[1]{\text{\rmfamily\upshape #1}}
% \usepackage[europeanresistors,americaninductors]{circuitikz}

\newcommand*{\const}{\ct{const}}
% Style to select only points from #1 to #2 (inclusive)
\pgfplotsset{select/.style 2 args={
    x filter/.code={
        \ifnum\coordindex<#1\def\pgfmathresult{}\fi
        \ifnum\coordindex>#2\def\pgfmathresult{}\fi
    }
}}
\usepackage{array}\begin{document}\begin{tikzpicture}[
	force/.style={>=latex,draw=blue,fill=blue},
	acceleration/.style={>=open triangle 60,draw=blue,fill=blue},
	% axis/.style={densely dashed,gray,font=\small},
	axis/.style={densely dashed,black!60,font=\small},
	M/.style={rectangle,draw,fill=lightgray,minimum size=0.5cm,thin},
	m2/.style={draw=black!30, rectangle,draw,thin, fill=blue!2, minimum width=0.7cm,minimum height=0.7cm},
	m1/.style={draw=black!30, rectangle,draw,thin, fill=blue!2, minimum width=0.7cm,minimum height=0.7cm},
	plane/.style={draw=black!30, very thick, fill=blue!5, line width=1pt},
	% base/.style={draw=black!70, very thick, fill=blue!4, line width=2pt},
	string/.style={draw=black, thick},
	pulley/.style={thick},
	interface1/.style={draw=gray!60,
		% The border decoration is a path replacing decorator. 
		% For the interface style we want to draw the original path.
		% The postaction option is therefore used to ensure that the
		% border decoration is drawn *after* the original path.
		postaction={draw=gray!60,decorate,decoration={border,angle=-135,
					amplitude=0.3cm,segment length=2mm}}},
	interface/.style={
		pattern = north east lines,
		draw    = none,
		pattern color=gray!60,          
	},
	plank/.style={
		fill=black!60, 
		draw=black,
		minimum width=3cm,
		inner ysep=0.1cm,
		outer sep=0pt,
		yshift=0.75cm,
		pattern = north east lines,
		pattern color=gray!60, 
	},
	cargo/.style={
		rectangle,
		fill=black!70,              
		inner sep=2.5mm,
	}
]
	\draw (0,0) arc(-90:90:3cm);
	\draw[dotted](0,0) arc(-90:-270:3cm);
	\draw (0,0) -- +(-5,0);
	\draw[dotted] (0,0) -- +(5,0);

	\coordinate (0) at (0,0);
	\coordinate (I) at (3,3);
	\coordinate (II) at (0,6);
	\coordinate (c) at (0,3);

	\draw[fill=black] (c) circle (1.25pt) (I) circle (1.25pt) (II) circle (1.25pt) (0) circle (1.25pt);
	\draw[axis] (c) -- (I) (c) -- (II) (c) -- (0);

	\draw[force,->] (0) -- ++(1,0) node[below] {$\vec{v}$};
	\draw[force,->] (I) -- ++(0,1) node[right] {$\vec{v}$};
	\draw[force,->] (II) -- ++(-1,0) node[above] {$\vec{v}$};

	\draw[force,->] (0) -- ++(0,-0.5) node[right] {$m\vec{g}$};
	\draw[force,->] (I) -- ++(0,-0.5) node[left] {$m\vec{g}$};
	\draw[force,->] (II) -- ++(0,-0.5) node[right] {$m\vec{g}$};

	\draw[force,->] (0) -- ++(0,1) node[right] {$\vec{N}_1$};
	\draw[force,axis,->] (I) -- ++(-1,0) node[below] {$\vec{N}_{2n}$};
	\draw[force,axis,->] (I) -- ++(0,0.5) node[right] {$\vec{N}_{2\tau}$};
	\draw[force,axis,->] (I) -- ++(-1,0.5) node[above] {$\vec{N}_{2}$};
	\draw[force,->] (II) -- ++(0,-1) node[right] {$\vec{N}_3$};

	% \draw[force, axis, ->] (0) -- ++(0,0.5) node[left] {$\vec{Q}_1$};
	% \draw[force, axis, ->] (I) -- ++(-1,-0.5) node[below] {$\vec{Q}_2$};
	% \draw[force, axis, ->] (II) -- ++(0,-1.5) node[right] {$\vec{Q}_3$};

	\draw[force, axis, ->] (0) -- ++(0,-1.5) node[left] {$\vec{P}_1$};
	\draw[force, axis, ->] (I) -- ++(1,-0.5) node[above] {$\vec{P}_2$};
	\draw[force, axis, ->] (II) -- ++(0,0.5) node[right] {$\vec{P}_3$};



\end{tikzpicture}\end{document}